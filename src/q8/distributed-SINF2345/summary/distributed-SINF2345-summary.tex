
\documentclass[en,license=none]{../../../eplsummary}
\usepackage{listings}

\usepackage{caption}
\renewcommand{\familydefault}{\sfdefault}
\lstdefinelanguage{diff}{
  morecomment=[f][\color{blue}]{@@},     % group identifier
  morecomment=[f][\color{red}]-,         % deleted lines 
  morecomment=[f][\color{green}]+,       % added lines
  morecomment=[f][\color{magenta}]{---}, % Diff header lines (must appear after +,-)
  morecomment=[f][\color{magenta}]{+++},
}
\lstdefinelanguage{scala}{
  morekeywords={abstract,case,catch,class,def,%
    do,else,extends,false,final,finally,%
    for,if,implicit,import,match,mixin,%
    new,null,object,override,package,%
    private,protected,requires,return,sealed,%
    super,this,throw,trait,true,try,%
    type,val,var,while,with,yield},
  otherkeywords={=>,<-,<\%,<:,>:,\#,@},
  sensitive=true,
  morecomment=[l]{//},
  morecomment=[n]{/*}{*/},
  morestring=[b]",
  morestring=[b]',
  morestring=[b]"""
}
\lstset{
	  language=scala,
	  frame=single,
	  flexiblecolumns=true,
	  numbers=none, % left
	  stepnumber=1,
	  numberstyle=\ttfamily\tiny,
	  keywordstyle=\ttfamily\textcolor{blue},
	  stringstyle=\ttfamily\textcolor{red},
	  commentstyle=\ttfamily\textcolor{green},
	  breaklines=true,
	  extendedchars=true,
	  basicstyle=\ttfamily\footnotesize,
	  showstringspaces=false
	}



\hypertitle{Distributed}{8}{INGI}{2345}
{Nicolas Houtain \and Gorby Nicolas Kabasele Ndonda}
{Peter Van Roy}
$$$$

Attention : This summary is actually based on course note


(Need slide to understand)

\section{Parallel and distributed computing}
A distributed system is a set of nodes,connected by a network
which appear to its users as a single coherent system.

\begin{itemize}
    \item Parallel computing : many node, optimize performance, no
        failure
    \item[$\to$] Tightly coupled(low latency/delay and high performance)

    \item Distributed computing : many node in collaboration with
        \textcolor{red}{partial failure}
    \item[$\to$] Loosely coupled(high latency and low perfomance)
\end{itemize}

\section{Core Problems}
\subsection{Consensus}
Consensus is the process of agreeing on a number.
Problem is that all the nodes propose a value and some nodes might
crash \& stop repsonding.

\subsection{Atomic Broadcast}
If a node broadcast a message, all nodes must deliver 
the message in the same order.

\subsection{Relation}
Atomic broadcast $\equiv$ consensus (proof slide 13)

It's possible to resolve consensus if we have atomic broadcast and vice-versa.
\begin{enumerate}
    \item broadcast $\to$ consensus : We take the first proposal as 
    messages are received in the same order.
    \item consensus $\to$ broadcast : The subject of the consensus is the order to take.
\end{enumerate}

Paxos est ce qui est le plus utilisé pour les consensus\footnote{\url{http://research.microsoft.com/en-us/um/people/lamport/pubs/paxos-simple.pdf}}
 %J'ai vu un TODO mais je suis pas sur qu'il faille connaitre.

\section{Concurrency Aspects}

\begin{itemize}
    \item Asynchronous : There is no bound on the time for a message to
     arrive and to be computed, it resolve consensus iff 0 node crashes
    \item Partially synchronous : It start asynchronous and then become
        synchronous(it get an upper bound, we know it will happen but we
        don't know when.)
	  Consensus up to $\frac{n}{2}$ crashes
    \item Synchronous : Bound known for delivering and computation of message. Consensus with n-1 crashes
\end{itemize}

\paragraph{Asynchronous vs Synchronous}

Bound is simulated with an expected bound to be in partially synchronous.

\section{Failure Aspects}
Each node use a failure detector that is implemented by 
heartbeat and waiting.\\
Problem $\to$ bound exist but we don't know the exact value because
this bound can change with time (if RTT increase for example), 
We need to adapt the bound.

Other kinds fault than crash can appears
\begin{itemize}
    \item \textbf{Byzantine faults} : Sending wrong information, omit
        messages,\ldots
        \begin{enumerate}
            \item[$\to$] Byzantine algorithm tolerate $1/3$ faulty node and
                non-byzantine only $1/2$
        \end{enumerate}
    \item \textbf{Self-stabilizing} : It's important to know that system
        can be in a \textit{legitimate} or an
        \textit{illegitimate} state.

        It's robust to failure and don't need initialization!

        \begin{enumerate}
            \item[Need] 
                \begin{enumerate}
                    \item Convergence = from any illegitimate state,
                        system can eventually goes to a legitimate state
                    \item Closure = if in legitimate state, it remains
                        in a legitimate state.
                \end{enumerate}
        \end{enumerate}
\end{itemize}
For example in a token ring algorithm:
\begin{itemize}
	\item Illegitimate state: 0,2,3... token.
	\item Legitimate state : only one token.
\end{itemize}

\section{Formal models of distributed system}

\subsection{Modeling}

\begin{itemize}
    \item Continuous model : described by differential equations
    \item \textbf{Discrete event models} : described by state transition systems
\end{itemize}

Modeling need to be : Complete, Correct and Concise!

\subsubsection{State transition system}
$STS \equiv$ a set of states + rule for transition function
+ set of initial states

\begin{enumerate} 
    \item[$\to$] like finite state machine but no input
\end{enumerate}

\begin{itemize}
    \item A \textbf{configuration} is a snapshot of state of all node

        $$ C =(q_0, q_1, q_2,\cdots, q_{n-1}) $$  where $q_i$ is state of node $p_i$.
\end{itemize}

\paragraph{Property}
Determinism, I/O and atomicity.



\subsubsection{Node}
Can send, receive messages and do local computations.

A state is define by triple $<l, O, s>$ :
\begin{itemize}
    \item $l$ : inbuffer set for each neighbor
    \item $O$ : outbuffer set for each neighbor
    \item $s$ : local state
\end{itemize}

\begin{figure}[h]
    \centering
    \includegraphics[width=10cm]{img/node.png}
    \caption{Example states}
\end{figure}



\paragraph{Working}
\begin{enumerate}
    \item Wait for message
    \item When receive message, do some computation and send message
    \item Goto 1.
\end{enumerate}

\paragraph{Events}
\begin{itemize}
    \item comp(i) : computation event at process i. 

        \textit{Apply transition function f on node i state}

    \item del(i, j, m) : delivery event of msg m from i to j

        \textit{Move m from outbuf of $p_i$ to inbuf p $p_j$}

\end{itemize}

\subsubsection{Transition functions}

Formally a transaction functions requires for any two
$f(<I_1,O_1,s_f1>)=<I_2,O_2,s_2> \ and \ f(<I_3,O_3,s_3>)=<I_4,O_4,s_4>$
\begin{itemize}
	\item $I_2=I_4 =<\emptyset,...,\emptyset$ (all inbufs are empty)
	\item if $I_2=I_4 and s_1=s_3$ then
	\begin{itemize}
		\item $s_2=s_4$ (don't observe channel)
		\item $O_1[i] \subseteq O_2[i] \ and \ O_3[i] \subseteq O_4[i]$ 
		(only add msg to outbuf)
		\item $O_2[i]-O_1[i] = O_4[i]-O_3[i]$ (don't observe channel)
	\end{itemize}
\end{itemize}


\subsubsection{Execution}
An execution is a infinite sequence of ``$config_0, event_1, config_1,
event_2, config_2,\cdots$''


\begin{itemize}
    \item[If] $event_k = comp(i)$ : $config_{k-1}$ change to $config_k$
        by applying $p_i$'s transition function on i's state in
        $config_{k-1}$
    \item[If] $event_k = del(i, j, m)$ : $config_{k-1}$ change to $config_k$
        by moving m from i's outbuf to j's inbuf
\end{itemize}


\subsubsection{Property}

\begin{itemize}
    \item For each comp(i) is associated a \textbf{transition} $(state_1, state_2, i)$
    \item Transition $(s_1, s_2, j)$ is \textbf{applicable} in
        configuration c if  accesible state of node $j$ in c is $s_1$ 
    \item del(i, j, m) \textbf{application} in configuration c if m is
        in outbuf for link i-j of node i in c
        
        %TODO EXAMPLE SLIDE 11?

    \item \begin{enumerate}
            \item if transition e=($s_1, s_2, i$) is applicable
            \item or if e=del($i, j, m$) is applicable
        \end{enumerate}
        to configuration c, then app(e,c) is the new configuration after
        the event comp(i) or del($i,j,m$)

\end{itemize}


\subsection{Asynchronous (Schedules) / Synchronous}
Processes are deterministic,
Non-determinism comes from asynchrony(messages take arbitrary time to 
be delivered, and the time to compute varies)\ldots\\
A \textbf{schedule} is the sequence of events. The event are the
one that determine the properties(del($i,j,m$) determines the message
asynchrony and comp($i$) determines the proccess speed.)
 so all non-determinism is embedded in schedule.

\begin{itemize}
	 \item Given the initial conf, the schedule determines the whole 
	 execution.
	 \item Not all schedules are allowed for initial conf. (some 
	 event may be impossible to occurs.
\end{itemize}

\subsection{Order of event}
The order in which two applicable computation events or
two applicable delivery events are executed is irrelevant!\\
The idea of the proof is that if you two differents comp
events $a$ and $b$( meaning on different node!) applicable in a
configuration then appliying $a$ first will not change the state of the
node related to $b$ (vice-versa). 
\paragraph{Note} It is only true for event that are not causally
related!
\subsection{Admissible execution (Fairness)}
An execution is admissible if:
\begin{itemize}
	\item Each process has infinite number of comp(i).
	\item Every message m sent is eventually del($i,j,m$).
\end{itemize}
The infinity property permit messages to wait arbitrary long times before
being delivered.
\subsection{Synchronous Systems}
The execution is partitionned into disjoints rounds.  A round consist of 
deliver event for all message in outbuf and one compute event on 
every process.

\subsubsection{Causal order $<_H$}
Causal order is \textbf{transitive}.

\begin{itemize}
    \item[$ a <_H b $]
    \item if a occurs befor b on the same process
    \item if a produces m and b delivers m
    \item if a delivers m and b consumes m
\end{itemize}

\paragraph{Concurrent}
a and b are concurrent, $a || b$, if not $a <_H b$ and not $b <_H a$

\subsection{Similarity of execution}
\begin{itemize}
	\item The view of $p_i$ in E, denoted E|$p_i$ is the subsequence 
	of executions E restricted to events and state $p_i$.
	\item 2 executions E,F are similar w.r.r if E|$p_i$ = F|$p_i$.
\end{itemize}
\paragraph{Computation Theorem}
\begin{itemize}
	\item Let E an execution ($c_0,e_1,c_1,e_2,...$) and V the 
	schedule of event ($e_1,e_2,e_3,..$) s.t app$(e_i,e_{i-1})=c_i$
	\item Let P be a permutation of V preserving casual order.
\end{itemize}
Then E is similar to the execution starting in $c_0$ with schedule P. 
\subparagraph{Notations}
similar execution of E,F is noted F\textasciitilde E.
\begin{description}
	\item[Computations or Equivalence class:] A class s.t all the elements
are similar to each other.
\end{description}
Computation theorem implies two importants results:
\begin{enumerate}
	\item There is no algorithm that can observe the order of the sequence
	of events for all executions. %TODO ADD PROOF (pas compris)
	\item Computation theorem does not hold in a model extended s.t each
	process read an hardware clock.%%TODO ADD PROOF (pas compris)
\end{enumerate}
\subsection{Clock}
A clock is used to tell locally if two events are causally related.

\subsubsection{Lamport Clock}
\begin{itemize}
    \item Each process has a local logical clock, t initially t=0.
        Node p piggyback (t, p) on every sent message.
    \item On each event :
    \begin{enumerate}
        \item $t = max(t, t_q) + 1$ : when p receives message with
            timestamp ($t_q, q$) (delivery from q)
        \item $t = t+1$ : for every transistion (comp)
    \end{enumerate}

    \item[$\to$] 
        \begin{itemize}
            \item $(t_p, q) < (t_q, q) IFF (t_p <t_q \vee (t_p = t_q \wedge p <
        q))$
\end{itemize}
\end{itemize}

Lamport logical clock guarantee that if $a <_H b$, then $t(a) < t(b)$

\subsubsection{Vector clock}
\begin{itemize}
    \item Each process has a local vector, $v_p$ of size n. Initially
        $\forall_i v_p[i]=0$

        Node p piggyback $v_p$ on every sent message.
    \item On each event :
    \begin{enumerate}
        \item $v_p[p] = v_p[p] + 1$
        \item $\forall_i : v_p[i] = max(v_p[i], v_q[i])$
    \end{enumerate}

\item[$\to$] \begin{itemize}
        \item $v_p \leq v_q$ iff $\forall_i : v_p[i] \leq v_q[i]$
        \item $v_p < v_q$ iff $v_p \leq v_q$ and $\exists i : v_p[i] <
            v_q[i]$
    \end{itemize}
\end{itemize}

Vector clock guarantee that if $v(a) < v(b)$ then $a <_H b$ but also if
$a <_H b$ then $v(a) < v(b)$

\subparagraph{Precisions}
Vector clock cannot be done with smaller vector than size n for n nodes
\begin{itemize}
	\item The relation $<_H$ is a partial order
	\item The relation < on Lamport is a total order
	\item The relation < on vector is a partial order
\end{itemize}

\subsection{Complexity}
Defined over the 
\begin{itemize}
	\item Number of messages used before terminating
	\item Time it takes to terminate
\end{itemize}
An algorithm has terminated when all states in a config. are terminated
states and there is no more messages in (in/out)bufs.\\
\paragraph{Time Complexity}
A message delay is at most 1 time unit while a computation events take
0 time units. A \textbf{timed execution} is an exectuition s.t
\begin{itemize}
	\item Time is associated with each comp(i) event
	\item First event happens at time 0
	\item Time can never decrease and strictly increases locally.
	\item Max time between comp(i) sending m and comp(j)
	consuming m is 1 time unit.
\end{itemize}
Time complexity is maximum time until termination for all 
admissible timed executions.

\section{Specification and implementation of distributed systems}

\subsection{Event based component model}
Each node models a sequential program. There is a global clock
and at each tick either a node takes a one of the following step
\begin{itemize}
	\item Computation step : Perfoms computation(local) or
	sends/receives one message to/from other nodes(global)
	\item Communication step: deliver a message.
\end{itemize}
There are different models for the delivery :
\begin{enumerate}
    \item Receive 1 msg and send 1msg (Guerraoui)
    \item At most receive 1 msg and send at most 1 msg to each neighboor (Lynch)
    \item Receive k msg and send at most 1 msg to each neighboor (Welch)
\end{enumerate}


\paragraph{ }
Each program consists of a set of \textbf{modules or component
specifications}

\subsubsection{Event}
\begin{lstlisting}[mathescape]
upon event <RequestEvent, attr$_1$, attr$_2$,\cdots> do
// local computation
trigger <ResponseEvent, attr$_3$, attr$_4$,\cdots>
\end{lstlisting}


There is three man type of event : \textbf{request, indications,
confirmation}


\subsubsection{Modules}

\begin{figure}[h]
    \centering
    \includegraphics[width=10cm]{img/module.png}
    \caption{Modules scheme}
\end{figure}

\begin{itemize}
    \item receive instruction :  upon event <delBcaast | src, [$data_1$,
        $data_2$,...] > do
    \item send instruction :
        trigger <sendBcast | dest, [data$_1$, data$_2$,...]>
\end{itemize}

\begin{figure}[h]
    \centering
    \includegraphics[width=10cm]{img/ex_broadcast.png}
    \caption{Example application uses a broadcast}
\end{figure}


\subsection{Specification of a service}

\begin{figure}[h]
    \begin{tabular}{cc}
        \includegraphics[width=8cm]{img/ex_inter1.png} & 
        \includegraphics[width=8cm]{img/ex_inter2.png} \\
        \includegraphics[width=8cm]{img/ex_inter3.png} & 
        \includegraphics[width=8cm]{img/ex_inter4.png} 
    \end{tabular}
    \caption{Interface example}
\end{figure}


\subsection{Property}

A property P is a function that takes an execution and returns
true/false. (\textit{P is a predicate})

\begin{center}
    \textit{“Any [property] can be expressed as the
conjunction of a safety property and a
liveness property”}
\end{center}

\begin{itemize}
    \item \textbf{Prefix} of a execution E is the first k (for
        some k>0) configurations and events of E
    \item \textbf{Extension} of a prefix P is any execution
        that has P as a prefix

    \item \textbf{Safety} : Properties that state that something bad \textcolor{red}{never}
        happens

        \begin{figure}[!h]
            \centering
            \includegraphics[width=6cm]{img/safety.png}
            \caption{Safety is false if}
        \end{figure}

        \begin{itemize}
            \item[Note:] safety can only be satisfied in infinite time and violated in
                finite time
        \end{itemize}

    \item \textbf{Liveness} : Properties that state that something good
        \textcolor{red}{eventually} happens


        \begin{figure}[!h]
            \centering
            \includegraphics[width=6cm]{img/liveness.png}
            \caption{Liveness is true if}
        \end{figure}


        \begin{itemize}
            \item[Note:] liveness can only be satisfied in finite time and violated in
                infinite time
        \end{itemize}
\end{itemize}


\subsection{Failure}

\subsubsection{Node}

Nodes that don’t fail in an execution are
correct. There are different way of failure :

\begin{itemize}
    \item \textbf{Crash-stop} : stops taking steps, stops sending/receiving msg

        \begin{itemize}
            \item[$\to$] Cannot recover this failure
        \end{itemize}

    \item \textbf{Omissions} : send (resp. receive) ommission. Formally, an
        event removing element from outbuf[i] (resp. inbuf[i])

    \item \textbf{Crash-recovery} : stops taking steps but receiving and
        sending msg.

        \begin{itemize}
            \item[$\to$] We can recover after crashing with special $<Recovery>$ event
                autmatically generated.

                In practice, restarting in initial recovery state or on
                the save state if we make some (expensive) storage on
                permanent storage device.
        \end{itemize}

        A node is faulty in an execution if it crashes and never
        recovers or crashes/recovers infinitely.

        A correct node may crash and recover.

    \item \textbf{Byzantine} : sending messages/updating its state
        not specified by its algorithm.

        \textit{may behave maliciously, attacking the system}

\end{itemize}

\begin{figure}[h]
    \begin{tabular}{m{8cm}m{8cm}}
        \includegraphics[width=8cm]{img/fault-tolerance.png}
        &
        \begin{itemize}
            \item If node use stable storage : crash-recovery = omission
            \item If node use volatile storage : crash-recovery extend omission
                with amnesia
        \end{itemize}
    \end{tabular}

    \caption{Fault tolerance}
\end{figure}

\subsubsection{Channel}

\begin{itemize}
    \item \textbf{Fair-loss links} : Channel delivers any message sent
        with non-zero probability

        \begin{lstlisting}[caption= Fair loss link interface]
Request: <flp2pSend | dest, m>
Indication: <flp2pDeliver | src, m>
        \end{lstlisting}

        \begin{enumerate}
            \item FL1. \textbf{Fair-loss} : If m is sent infinitely
                often by $p_i$
                to $p_j$ , and neither crash, then m is delivered infinitely
                often by $p_j$
            \item FL2. \textbf{Finite duplication} : If a m is sent a finite
                number of times by $p_i$ to $p_j$ , then it is delivered a
                finite number of times by $p_j$
            \item FL3. \textbf{No creation} : No message is delivered unless it
                was sent
        \end{enumerate}

    \item \textbf{Stubborn links} : Channel delivers any message sent
        infinitely many times (to implement it,
        a list of all previously sent msg is kept, after a timeout all 
        the message will be resent)

        \begin{lstlisting}[caption= Stubborn link interface, mathescape]
upon event <Init> do
    sent := $\empty$
    startTimer(delay)

upon event <Timedout> do
    forall (dst, m) $\in$ sent do
        trigger <flp2pSend | dst, m>
    startTimer(delay)

// Request
upon event <sp2pSend | dest, m> do 
    send := sent $\cup$ { (dest, m) }

// Indication
upon evetn <sp2pDeliver src, m> do
    trigger <sp2pDeliver src, m>
        \end{lstlisting}

        \begin{enumerate}
            \item[SL1] \textbf{Stubborn delivery} : if a node $p_i$ sends a
                message m to a correct node $p_j$ , and $p_i$ does not
                crash, then $p_j$ delivers m an infinite number of
                times

            \item[SL2] \textbf{No creation} : if a message m is delivered by
                some node $p_j$ , then m was previously sent by
                some node $p_i$
        \end{enumerate}

    \item \textbf{Perfect links} : Channel that delivers any message
        sent exactly once (keep a set of delivered msg)

        \begin{lstlisting}[caption=Perfect links, mathescape]
upon event <Init> do delivered := $\empty$

upon event <pp2pDeliver | dst, m> do 
    trigger <sp2pSend dst, m>

upon event <sp2pDeliver | src, m> do
    if m $\notin$ delivered then
        delivered := delivered $\cup$ {m}
        trigger <pp2pDeliver | src, m>
        \end{lstlisting}

        \begin{enumerate}
            \item[PL1] \textbf{Reliable delivery} (liveness) : 
                If neither $p_i$ nor $p_j$ crashes, then every message sent
                by $p_i$ to $p_j$ is eventually delivered by $p_j$

            \item[PL2] \textbf{No duplication} (safety) : Every message is delivered
                at most once

            \item[PL3] \textbf{No creation} (safety) : No message is delivered unless it was
                sent
        \end{enumerate}

\end{itemize}

\subsection{Timing assumptions}

Different processing speeds of nodes and different speeds of messages.

\subsubsection{Local Vs Global}
\begin{itemize}
    \item Local (one node = State)
        \begin{itemize}
            \item Atomic
            \item Deterministic
        \end{itemize}
    \item Global (many node = Configuration)
        \begin{itemize}
            \item Non-atomic (because piece of code in many node)
            \item Non deterministic (because network and reveiv order message)
        \end{itemize}
\end{itemize}


\subsubsection{Synchronous Vs Asynchronous}
\begin{itemize}
    \item Asynchronous : No timing assumtion on nodes and channels.

        \begin{itemize}
           \item Lamport clocks (or vector clocks) to observe causality. 
           \item Total order not observable
       \end{itemize}

       Internet is asynchronous!

    \item Synchronous : Use round to synchronous (like clock) and this
        is use to detect failure

    \item Partial synchonous : asynchonous system wich eventually
        becomes synchronous.

        \textit{It’s just a way to formalize the following : Your
            algorithm will have a long enough time
            window, where everything behaves nicely
        (synchrony), so that it can achieve its goal}
\end{itemize}


\begin{figure}[h]
    \centering
    \includegraphics[width=8cm]{img/partialsynchronous.png}
    \caption{Partial synchony}
\end{figure}


\subsection{Failure detectors}

Use failure detectors to encapsulate timing assumtions.
Need completeness and accuracy.

\subsubsection{Typical implementation}
\begin{enumerate}
    \item Periodically exchange \textbf{heartbeat} messages
    \item Timeout based on worst case RTT
    \item if timeout, then suspect node
    \item if rcv msg from suspected node then revise suspicion and
        increase time-out
\end{enumerate}

\subsubsection{Modeling}

\begin{itemize}
    \item Configuration = state of each node + \textcolor{red}{FD\_state of each
        node}
    \item Transition function on node i gets extra parameter :
        \textcolor{red}{FD\_state of node i}
    \item FD\_state updated in comp(i) by \textcolor{red}{FD\_function}
\end{itemize}

\paragraph{Requirement}
\begin{itemize}
    \item Completeness
    \begin{enumerate}
        \item Strong : Every crashed node is eventually detected by all
            correct nodes

            \textit{here exists a time after which all crashed
            nodes are detected by all correct nodes}

        \item Weak : Every crashed node is eventually detected by some
            correct node

            \textit{There exists a time after which all crashed
            nodes are detected by some correct node}
    \end{enumerate}
    \item Accuracy
        \begin{enumerate}
            \item Strong : No correct node is ever suspected

            \item Weak : There exists a correct node which is never
                suspected by any node

            \item Eventual Strong accuracy : After some finite time the
                FD provides strong accuracy
            \item Eventual Weak accuracy : After some finite time the FD
                provides weak accuracy
        \end{enumerate}
\end{itemize}

\subsubsection{Different established detectors}

\begin{table}
    \begin{tabular}{c|c|c|c|c}
        & & \multicolumn{2}{c}{ Completeness} & \\
        & & Strong & Weak & Order\\
        \hline

        \multirow{2}{*}{Synchm} & Strong accuracy & Perfect detector (P) &
        Detector (Q) & (1) \\
        & Weak accuracy & Strong detector (S) & Weak detector (W) & (2) \\

        \hline

        \multirow{2}{*}{Asyn} & Strong accuracy & Eventually perfect detector
        ($<>P$) & Eventually detector Q ($<>q$) & (3) \\
        & Weak accuracy & Eventually strong detector ($<>S$) & Eventually weak detector
        ($<>W$) & (4) \\
    \end{tabular}
    \caption{$4 \preceq 3, \quad 4 \preceq 2, \quad 2 \preceq 1, \quad 3
    \preceq 1$}
\end{table}

Weak and strong are equivalent. (Weak equivalence $\preceq$ Strong
equivalence is trivial. L'inverse est effectué gràce à un broadcast des
suspect node)


\paragraph{Perfect detector}:

\begin{lstlisting}[caption = Perfect detector, mathescape]
indication <crash | $p_i$>
\end{lstlisting}

\begin{enumerate}
    \item Each node every $\gamma$ time : \texttt{Send <heardbeat>} to all nodes
    \item If don't receive \texttt{<heartbeat>} from node $p_i$ after $\phi + \gamma$ time :
    \texttt{Detect <crash | $p_i$}
\end{enumerate}

$\Rightarrow$ Synchronous system

\paragraph{Eventually perfect detector}:

\begin{lstlisting}[caption = Perfect detector, mathescape]
indication <suspect | $p_i$>
indication <restore | $p_i$>
\end{lstlisting}

\begin{enumerate}
    \item Each node every $\gamma$ time : \texttt{Send <heardbeat>} to all nodes
    \item If don't receive \texttt{<heartbeat>} from node $p_i$ after $T$ time :
    \texttt{Send <suspect | $p_i$} and add $p_i$ in suspected set
    \item If receive \texttt{<heartbeat>} from node $\in$ suspected :
    \texttt{<restore | $p_i$>} and remove from suspected
\end{enumerate}

$\Rightarrow$ Partially synchronous system

\subsection{Leader election}

Note, leader election is a FD : always suspects all nodes except one
(leader).

\paragraph{Which node ?}
Thi lower ID or better the lowest number of crash.

\paragraph{Property}
\begin{itemize}
            \item Completeness : eventually every correct node trusts
            some correct node
            \item Accuracy : No two correct nodes trust different correct nodes
        \end{itemize}

\begin{itemize}
    \item Leader election (LE) which matches P
        \begin{itemize}
            \item Eventual completeness : Eventually every correct
            node trusts some correct node. (detects failure)
            \item agreement : No two correct nodes different correct
            nodes.
            \item Local accuracy (if a node is elected leader by p i,
            all previously elected leaders by p i have crashed)
        \end{itemize}
    \item Eventual leader election (LE) which matches $<>$P
        \begin{itemize}
            \item Eventual completeness (detects failure)
            \item Eventual agreement 
        \end{itemize}

\end{itemize}

\subsubsection{Leader election}
\begin{lstlisting}[caption = leader election with perfect failure detector, mathescape]
Indication: <leLeader : $p_i$>

upon event <Init> do 
    suspected = $\empty$
    leader := highest(r)
    trigger <leLeader | leader>

upon event <crash | $p_i$> do
    suspected := suspected $\cup$ {$p_i$}

upon exists $p_i$ s.t. r($p_i$) $\subseteq$ suspected $\wedge$ $p_i \notin$ suspected do
    leader := $p_i$
    trigger <leLeader | $p_i$>
\end{lstlisting}

\begin{enumerate}
    \item[LE1] \textbf{eventual completenesse} : Eventually every correct node
    trusts some correct node
    \item[LE2] \textbf{agreement} : not two correct nodes trust different correct
    nodes
    \item[LE3] \textbf{local accuracy} : is a node is elected leader by $p_i$,
    all previosly elected leaders by $p_i$ have crashed
    \end{enumerate}

\subsubsection{Eventual leader election}
\begin{lstlisting}[caption = eventually leader election, mathescape]
Indication: <leLeader : $p_i$>

upon event <Init> do 
    suspected = $\empty$
    leader := $p_n$
    trigger <leLeader | leader>

upon event <suspect | $p_i$> do
    suspected := suspected $\cup$ {$p_i$}

upon event <restore | $p_i$> do
    suspected := suspected \\ {$p_i$}

upon exists $p_i$ s.t. r($p_i$) $\subseteq$ suspected $\wedge$ $p_i \notin$ suspected do
    leader := $p_i$
    trigger <leLeader | $p_i$>
\end{lstlisting}

\begin{enumerate}
    \item[LE1] \textbf{eventual completenesse} : Eventually every correct node
    trusts some correct node
    \item[LE2] \textbf{eventual agreement} : Eventually not two correct nodes trust different correct
    nodes
\end{enumerate}


\subsection{Reductions}
We say X $\preceq$ Y (X is reducible to Y)
 if X can be solved given a solution of Y.
 
\paragraph{Preorder $\preceq$} :
\begin{itemize}
    \item Reflexivity
    \item Transitivity
\end{itemize}
It is not antisymmetric, thus it's not a partial order.
 We say that X $\simeq$ Y (equivalent) if X $\preceq$ Y and Y $\preceq$ X.
\paragraph{Partial order} :
Preorder + antisymmetric

\section{Reliable broadcast}

\subsection{Best-effort broadcast}

\begin{itemize}
	\item Best-effort-Validity : if $p_i$ and $p_j$ are correct,
	then any broadcast by $p_i$ is eventually delivered by $p_j$
	\item No duplication : No message delivered  more than once
	\item No creation : No message delivered unless broadcast.
\end{itemize}

BeB gives no guarantee if the sender crash. 

\begin{lstlisting}[caption=Beb, mathescape]
upon event <BebBroadcast | m> do
    for all $p \in dst$
        trigger <BebDeliver | src, m>
\end{lstlisting}

Implementation is straightforward , simple loop on all nodes.

\subsection{Reliable Broadcast}
\begin{itemize}
	\item Validity: same as BeB
	\item No duplication.
	\item No creation.
	\item Agreement : if a correct node delivers m, then every correct
	node delivers m.
\end{itemize}
\subsubsection{Uniform Reliable Broadcast}
If a failed node delivers, everyone must delivers.
It add the Uniform agreement: For any message m, 
if a process delivers m, then every correct process delivers m.

\subsubsection{Implementation}
\textbf{Using Beb and P}

\begin{lstlisting}[mathescape, caption= Lazy reliable broadcast]
upon event <Init> do 
    delivered = $\empty$
    correct := $\pi$
    forall $p_i \in \pi$ do from[$p_i$] := $\empty$

upon event <rbBroadcast | m> do
    trigger <bebBroadcast | (DATA, self, m)>

upon event <crash | $p_i$> do
    correct := correct \ {$p_i$}
    forall ($s_m$, m) $\in$ from[$p_i$] do
        trigger <bebBroadcast | (DATA, $s_m$, m)>

upon event <bebDeliver | $p_i$, (DATA, $s_m$, m)> do
    if m $\notin$ delivered then
        delivered += {m}
        from[$p_i$] += ($s_m, m$) 
        trigger <rbDeliver | $s_m$, m>
        if $p_i \notin$ correct then
            trigger <bebBroadcast | (DATA, $s_m$, m)>
\end{lstlisting}

\paragraph{Proof correctness}
If correct $p_j$ delivers msg broadcast by $p_i$
\begin{itemize}
	\item If $p_i$ is correct, BeB ensures correct delivery
	\item If $p_i$ crashes,
		\begin{itemize}
			\item $p_j$ detects this (completeness).
			\item $p_j$ uses BeB to ensure every correct node gets it.
		\end{itemize}
\end{itemize}
If we use a eventually perfect detector, it only affects 
performance, not correctness.


\subsection{Eager RB}
The exists a eager version of reliable broadcast that does not use a 
perfect detector. It simply use a BeB broadcast upon receive of a 
message.


\begin{lstlisting}[mathescape, caption= Eager reliable broadcast]
upon event <Init> do 
    delivered = $\empty$

upon event <rbBroadcast | m> do
    delivered += {m}
    trigger <rbDelivered  | self, m>
    trigger <bebBroadcast | (DATA, self, m)>

upon event <bebDeliver | $p_i$, (DATA, $s_m$, m)> do
    if m $\notin$ delivered then
        delivered += {m}
        trigger <rbDeliver | $s_m$, m>
        trigger <bebBroadcast | (DATA, $s_m$, m)>
\end{lstlisting}


\subsubsection{Correctness of Eager RB}
if correct $p_j$ delivers message bcast by $p_i$,
$p_j$ uses BeB to ensure every correct node gets it. 


\subsection{Uniformity RB}
In the above version, uniformity is not respected. If a sender
p immediately RB delivers and crashes, only p will have delivered 
the message.

\begin{lstlisting}[mathescape, caption= Uniformity reliable broadcast]
upon event <urbBroadcast | m> do
    pending += {(self, m)}
    trigger <bebBroadcast | (DATA, self, m)>

upon event <bebDeliver | $p_i$, (DATA, $s_m$, m)> do
    ack[m] += {$p_i$}
    if ($s_m$, m) $\notin$ pending then
        pending += {($s_m$, m)}
        trigger <bebBroadcast | (DATA, $s_m$, m)>

upon exist ($s_m$, m) $\in$ pending s.t canDeliverd(m) AND m $\notin$ delivered do
    delivered += {m}
    trigger <urbDeliver | $s_m$, m>
\end{lstlisting}


The idea is the following, message are pending until all correct
nodes get it. Node that gets the message exchange ack. The 
message is deliver once all the nodes acked.

%TODO CORRECTNESS AND MAJORITY ACK
\subsubsection{Correctness}
\begin{itemize}
    \item No creation from BEB
    \item No duplication by using delivered set
    \item TODO
\end{itemize}

\section{Causal-Order Broadcast}
Uniform reliable broadcast doesn't deal with the order in which the
message are deliver. The causal-order broadcast remedy to this 
problem.
\paragraph{}
We that $m_1 \to m_2$ ($m_1$ causally precedes $m_2$) if
\begin{itemize}
    \item FIFO order: Some process $p_i$ broadcasts $m_1$ before
        broadcasting $m_2$
    \item Network order: Some process $p_i$ delivers $m_1$ and
        later broadcasts $m_2$
    \item Transitivity: There is a message m' s.t $m_1 \to m' \ and \ m'\to m_2$
\end{itemize}
\paragraph{Property}
\begin{itemize}
    \item \textbf{CB:} If node $p_i$ delivers $m_1$, then $p_i$ must have 
        delivered every message causally preceding ($\to$) $m_1$ before $m_1$.
    \item \textbf{CB':} If $p_j$ delivers $m_1$ and $m_2$,and $m_1\to m_2$
        then $p_j$ must deliver $m_1$ berfore $m_2$.
\end{itemize}
\begin{figure}[h]
    \centering
    \includegraphics[scale=0.8]{img/cbs_properties.png}
\end{figure}
\subsection{Implementation}
The idea is to broadcast the message along with its history( message 
that were sent before it). This history is an ordered list of
causally preceding messages called $past_m$

\begin{lstlisting}[caption=Causal broadcast, mathescape]
upon event <Init> do
    deliverd := $\empty$
    past := nil

upon event <rcoBroadcast | m> do
    trigger <rbBroadcast | (DATA, past, m)>
    past := append(past, <$p_i$, m>)

upon event <rbDeliver | $p_i$, (DATA, past$_m$, m)> do
    if m $\notin$ delivered then
        forall ($s_n$, n) $\in$ past$_m$ do
            if n $\notin$ delivered then
                trigger <rcoDeliver | $s_n$, n>
                delivered += {n}
                past := append(past, <$s_n$, n>)
        
        trigger <rcoDeliver | $p_i$, m>
        delivered += {m}
        past := append(past, <$p_i$, m>)
        
\end{lstlisting}

\subsubsection{First algorithm}
The problem with this algorithm is that the size of the message
grows. The idea to improve the algorithm is to detect  with 
ack when all correct node got the message and delete from
past if it's the case (Garbage Collector).


\begin{lstlisting}[caption=GC causal broadcast, mathescape]
upon event <Init> do
    deliverd := $\empty$
    correct := $\pi$
    forall m : ack[m] := $\empty$

upon event <crash | $p_i$> do
    correct := correct \ {$p_i$}

upon event m $\in$ deliverd AND seld $\notin$ ack[m] do
    ack := ack[m] $\cup$ {self}
    trigger <rbBroadcast | (ACK, m)>

upon event <rbDeliver | $p_i$, [ACK, m]> do
    ack := ack[m] $\cup$ {$p_i$}
    if correct $\subseteq$ ack[m] do
        past := remove (past, <x, m>)
\end{lstlisting}
%TODO Question on GC

\subsubsection{Second algorithm}
In the first algortihm, the history was a list, in the second its a
\textbf{vector timestamp}(vector clock).
Each node has a vector clock s.t a node $p_i$:
\begin{itemize}
    \item VC[i] : number of messages $p_i$ coBroadcasted
    \item VC[j], j $\neq$i: number of messages $p_i$ coDelivered from $p_j$
\end{itemize}
The delivery of m is only done if $VC_m$(attached VC) precedes $VC_i$ 
(local clock)

\begin{lstlisting}[caption=VC causal broadcast, mathescape]
upon event <Init> do
    forall $p_i \in \pi$ do VC[i] := 0

upon event <rcoBroadcast | m> do
    trigger <rbBroadcast | (DATA, VC, m)>
    VC[self]++
    trigger <rcoDeliver | self, m>

upon event <rbDeliver | $p_i$, (DATA, VC$_m$, m)> do
    if $p_i$ != self then
        pending += ($p_i$, (DATA, VC$_m$, m))
        delivered-pending()

procedure delivered-pending()
    while exists x=($s_m$, (DATA, VC$_m$, m)) $\in$ pending s.t VC $\ge$ VC$_m$ do
        pending := pending | ($s_m$, (DATA, VC$_m$, m))
        trigger <rcoDeliver | $s_m$, m>
        VC[rank($s_m$)]++
\end{lstlisting}


\subsection{Different Possible Orderings}
\subsubsection{FIFO order}
\begin{itemize}
    \item Message form same node delivered in order sent.
    \item For all messages $m_1$ and $m_2$ and all $p_i$ $p_j$:
        \begin{itemize}
            \item if $p_i$ broadcasts $m_1$ before $m_2$, and 
                if $p_j$ delivers $m_1$ and $m_2$, then $p_j$ delivers
                $m_1$ before $m_2$.
        \end{itemize}
\end{itemize}
This definition doesn't require the delivery if both messages.
\subsubsection{Total order}
\begin{itemize}
    \item Everyone delivers everythings in exact same order.
    \item For all messages $m_1$ and $m_2$ and all $p_i$ $p_j$:
        \begin{itemize}
            \item if  both $p_i$ and $p_j$  delivers both messages
                then they deliver them in the same order.
        \end{itemize}
\end{itemize}
The order is not necessarily the one in which the message where sent.
Does not require the delivery of both message 
\subsubsection{Hierarchy of Orderings}
\begin{figure}[h]
    \centering
    \includegraphics[scale=0.7]{img/hierarchy_ord.png}
\end{figure}





\section{Consensus}

Nodes proposes value and they must agree on one of these values.

\paragraph{Key problem} Solve total order broadcast, atomic commit, reliable broadcast


\subsection{Property}
\begin{description}
	\item[Validity] Any value decided is a value proposed
	\item[Agreement] No two correct nodes decide differently 
	(in a \textbf{uniform consensus}, node does not need to be correct)
	\item[Termination] Every correct node eventually decides
	\item[intergrity] A node decides at most once
\end{description}

\begin{lstlisting}[caption=Consensus interface]
Request: <cPropose | v>
Indication: <cDecice | v>
\end{lstlisting}


\subsection{Hierachical Consensus}

\begin{itemize}

    \item Use perfect P and BEB

    \item Each nodes stores its \texttt{proposal} and the identifier of the  last adopted 
        proposer in \texttt{lastprop}.
    
    \item Loop throug round 1 to N

        \begin{tabular}{m{1.3cm}m{13cm}}
            Round i&
        \begin{itemize}
            \item node i is leader and broadcasts and decide its proposal  v
            \item Other nodes adopt i proposal v (save it in lastprop) or 
                detect crash of i.

            \item[$\to$] Future round will only propose v
        \end{itemize}
    \end{tabular}
\end{itemize}

\subsubsection{Orphan message issue}

Problem is that broadcast can be delayed and the leader crash
and a node can receive two proposal in the next round. It will
therefore affect future round.

\begin{figure}[!h]
    \centering
    \includegraphics[width=8cm]{img/orphan.png}
    \caption{Orphan}
\end{figure}

\paragraph{Solution}

Counter measure $\to$ ranking of the node based on there identifier
($p_1>p_2>p_3>...$)

$\Rightarrow$ Adopt if proposer p is ranked lower than lastprop otherwise p has
crashed and should be ignored

\subsubsection{Implementation}
\begin{lstlisting}[caption=Hierachical consensus, mathescape]
upong event <Init> do
    detected :=  $\empty$
    round    := 1
    proposal := $\top$
    lastprop := 0

    for i=1 to N do
        broadcast[i] := delivered[i] := false

upon event <Crash | p$_i$ do 
    detected := detected $\cup$ {rand(p$_i$)}

upon event <cPropose | v> do
    if proposal = $\top$ then
        proposal := v

upon round = rand(self) and
     broadcast[round] = false and
     proposal != $\top$ do

    broadcaset[round] := true
    trigger <cDecide | proposal>
    trigger <bebBroadcast | (DECIDED, round, proposal)>

upond event <bebDeliver | p$_i$, (DECIDED, r, v)> do
    if r > lastprop then
        proposal := v; lastprop := r
    delivered[r] := true

upon delivered[round] or round $\in$ detected do
    round := round + 1
\end{lstlisting}


\subsubsection{Correctess}
\begin{itemize}
	\item Validity? $\to$ Always decide own proposal or adopted value
	\item Integrity? $\to$ Rounds increase monotonically, node only 
	decide when leader.
	\item Termination? $\to$ Every correct node makes it to the round it 
	is leader in:
	\begin{itemize}
		\item If some leader fails, ccompleteness of P ensure progress.
		\item If leader correct, validity of BeB ensures delivery. 
	\end{itemize}
	\item Agreement? $\to$ take correct leader with minimum id i.
        \begin{itemize}
            \item By termination it will decide v and v will be BEB
            \item[$\to$] Every correct node adopts v no older proposal 
                can override the adoption
        \end{itemize}
\end{itemize}

\textbf{Failute-torerant up to N-1} 

\subsubsection{Formalism}
\begin{lstlisting}[caption=Hierarchical consensus]
x i := proposal
for r:=1 to N do
    if r=i then
        forall j in 1..N do 
            send <val, x i , r> to P j ;
        decide x i
    if collect <val, x', r> from r then
        x i := x';
end
\end{lstlisting}

\subsection{Uniform consensus}
Compare to nonuniform consensus where node i decides in round i,in
uniform consensus with P, node move the decision to the end.

\begin{lstlisting}[caption=Hierarchical consensus]
x i := proposal
for r:=1 to N do
    if r=i then
        forall j in 1..N do 
            send <val, x i , r> to P j ;
    if collect <val, x', r> from r then
        x i := x';
end
decide x i
\end{lstlisting}

\subsubsection{With inaccurate FD}

\begin{figure}[h]
    \centering
    \begin{tabular}{m{8cm}m{8cm}}
        \includegraphics[width=8cm]{img/hc_fd.png} &
        \includegraphics[width=8cm]{img/uhc_fd.png} \\
        p2 suspect p1, p3 suspect p2 (nonuniform consensus) &
        p2 suspect p1, p3 suspect p2, p1 suspect p3 (uniform consensus)\\
        \multicolumn{2}{c}{$\Rightarrow$ Does'nt work with innacurrate FD}
    \end{tabular}
        \caption{Inaccuracy with failure detector}
\end{figure}

\textbf{The algorithm doesn't work with innaccurate FD}

\subsubsection{With Strong failure detectore}

The algorithm works with a S (strong dectector). The difference between
Strong detector (weaker than P) and a Perfect detector comes from the
accuracy w.r.t one node.

\paragraph{Correctess}
\begin{itemize}
	\item Validity? $\to$ Always decide own proposal or adopted value
	\item Integrity? $\to$ Rounds increase monotonically, node only 
	decide when leader.
	\item Termination? $\to$ Every correct node makes it to the last round
	\begin{itemize}
		\item If some leader fails, completeness of S ensure progress.
		\item If leader correct, validity of BeB ensures delivery. 
	\end{itemize}
	\item Uniform agreement? $\to$ take an accurate correct leader with id i.
        \begin{itemize}
            \item By weak accuracy (S) and termination such a node exists
                and it will BEB v
        \end{itemize}
\end{itemize}


\subsubsection{With Eventual failure detector}
Evenutally perfect detector cannot solve consensus with resilience t $\geq$ n/2\\

%Proof?
\begin{itemize}
	\item Rotating coordinator from before will not work because
	''eventually'' might be after the first N rounds
	\item The idea is to rotate forever and eventually all nodes
	correct w.r.t 1 coordinator (coordinator's value becomes
	agreed value)
\end{itemize}

\paragraph{Termination}
There will be a bound on the number of failures $\to$ less (f<n/3) than a third
can fail. This will allow to decide using majority.

\begin{enumerate}
	\item Everyone send vote to coordinator C
	\item C picks majority vote V, and broadcasts V
	\item Every node get broadcast, change vote to V
	\item Change coordinator C and go to 1
\end{enumerate}

\begin{lstlisting}[caption=rotate coordinator for <>S]
x i := input
r=0
while true do
begin
    r:=r+1                          
    c:=(r mod n)+1
    send <value, x i , r> to p c


    if i==c then
    begin
        msgs[0]:=0; msgs[1]:=0;
        for x:=1 to n-f do
        begin
            receive <value, V, R> from q
            msgs[V]++;
        end
        if msgs[0]>msgs[1] then v:=0 else v:=1 end
        if msgs[0]&=& 0 or msgs[1]&=& 0 then d:=1 else d:=0 end 
        forall j do send <outcome, v, r> to p j
    end


    if collect<outcome, v, r> from p c then
    begin
        x i := v
        
        if d and i!=0 then begin decide(v); i:=0; end 
    end
end
\end{lstlisting}

\begin{itemize}
    \item If at least n-f nodes vote V in round r, every leader will see a majority for V
        in all rounds > r. 

        \paragraph{Proof}:
        \begin{itemize}
            \item We know that at most f nodes don't vote V
            \item We also know n/3<(n-f)/2 (because f<n/3 implies n-f>2n/3)
                $\to$ f<(n-f)/2 (because f<n/3 and n/3<(n-f)/2)
            \item So less than half of any n-f nodes do not vote V
        \end{itemize}
\end{itemize}


\section{Terminating Reliable Broadcast}

With normal reliable broadcast, no idea when or if a message will be delivered.

$\Rightarrow$ In TRB, sender broadcast M and receiver await delivery M. All nodes
either deliver M or ''abort'' (<SF> sender Faulty message)

TRB requires synchrony!

\begin{lstlisting}
Request: <trbBroadcast | src, m>

Request: <trbBroadcast | src, m>
\end{lstlisting}

\subsection{Property}
\begin{description}
	\item[Termination] Every correct node eventually delivers one message
	\item[Validity] If correct src sends m, then src will deliver m
	\item[Uniform agreement] If any node delivers m, then every correct node
	eventually delivers m
	\item[Intergrity] If a node delivers m, then either m=<SF> or m was
	broadcast by src.
\end{description}

\subsection{Consensus based TRB}
Src RB broadcast m (deliver <SF> if src is suspected by P)
\begin{itemize}
	\item Src BeB broadcast m
	\item Nodes propose whichever comes first: crash suspicion(<SF>) or 
	BeB delivery from src (M)
	\item Deliver consensus decision
\end{itemize}


\subsubsection{Hardness}
\begin{itemize}
    \item $consensus \prec TRB$ $\Rightarrow$ can't have a asynchronous networks for TRB.
    \item $ P \prec TRB \Rightarrow$ can't have a eventually synchonous systems
\end{itemize}

\subsubsection{Correctness}
TODO

\section{Total order broadcast (consensus)}

The order imposed by causal broadcast is partial. Some message might 
be delivered in different order by different process.

With \textbf{total order} broadcast, process must deliver all
messages according to the same order.

\paragraph{\textbf{Total order broadcast} $\equiv$ \textbf{consensus}}

\subsection{Specification}
The specification of the total order broadcast are roughly the same as the reliable broadcast
there only a little modification

\begin{description}
	\item[Uniform agreement] For any message m: if any process delivers
	m, then every correct process delivers m.
\end{description}
\
\subsection{Type of total order}

\begin{lstlisting}
Request: <toBroadcast, m>
Indication: <toDeliver, src, m>
\end{lstlisting}

\subsubsection{Total order}
\begin{itemize}
	\item Let m1 and m2 be any two messages and let pi and pj be any two 
	correct processes that deliver m1 and m2
	\item If pi delivers m1 before m2, then pj delivers m1 before m2
\end{itemize}

\subsubsection{Uniform total order}
\begin{itemize}
	\item Let m1 and m2 be any two messages and let pi and pj be any two
	processess that deliver m2 (only m2!).
	\item If pi delivers m1 before m2, then pj delivers m1 before m2
\end{itemize}

\subsection{Consensus based algorithm}
\begin{lstlisting}[caption=Total order implementation, mathescape]
upon event <Init> do
    unordered = delivered = $\empty$
    wait := false
    sn := 1

upon event <toBroadcast, m> do
    trigger <rbBroadcast, m>

upon event <rbDeliver, sm, m> AND (m $\notin$ delivered) do
    unordered += {(sm, m)}

upon (unordored != $\empty$) AND not(wait) do
    wait := true
    trigger <Propose sn, unordered>

upon event <Decide sn, decided> do
    unordered := unordered \ decided
    ordered = deterministicSort(decided)

    for all (sm, m) $\in$ ordered:
        trigger <toDeliver, sm, m>
        delivered += {m}

    sn++
    wait := false
\end{lstlisting}


\section{Shared Memory}

In real shared memory there is no message-passing, node access one
shared memory. But in distributed system we simulate shared memory 
using message passing.

\begin{itemize}
	\item A register represents each memory location (objects)
    \item Node can \texttt{read(R) => x}/\texttt{write(R, x)}
	\item Simplification of key-value stores
\end{itemize}

\paragraph{Basic Assumptions}
\begin{itemize}
    \item Nodes are sequential, they can only do one operation at a time. 
invocation,response,invocation,response,\ldots
\item The values are positive integers initially zero.
    \end{itemize}

\paragraph{Definitions}
In an execution, an operation is 
\begin{itemize}
	\item Complete if both invocation \& response occured
	\item Failed if invoked, but no response arrives
\end{itemize}

$op_1$ precedes $op_2$ if (denoted $<_p$) if response of $op_1$ 
precedes invocation of $op_2$. Otherwise they are concurrent.

\paragraph{Terminology}
\begin{itemize}
	\item (1,N)-algorithm : 1 designated writer, multiple readers
	\item (N,N)-algorithm : Multiple writers, multiple readers
\end{itemize}


\subsection{Regular Register (1,N)}
\begin{itemize}
	\item Termination : Each read and write operation of a correct node 
	completes.
	\item Validity : Read return last value written if
		\begin{itemize}
			\item Read is not concurrent with another write, and 
			\item Read is not concurrent with a failed operation
		\end{itemize}
	Otherwisz the read must return the last value 
	written or a concurrent value being written.
\end{itemize}


\subsubsection{Centralized Algorithm}


One process is designated as leader. For reading, the latest value must
be ask to the leader. For writing, leader value is updated. 

$\Rightarrow$ does not work if leader crashes.

\subsubsection{Bogus Algorithm}
\begin{itemize}
    \item[read] : return local value
    \item[write] : overwrite local value and broadcast change
\end{itemize}

\begin{figure}[h]
    \centering
        \includegraphics[width=6cm]{img/bogus_1.png}
        \caption{Bogus algorithm}
\end{figure}


\paragraph{Read-one write all (1, N)}
Bogus algorithm modified can be use with a \textbf{perfect FD}
and fail-stop model.

\begin{itemize}
    \item[$\to$] Write need to wait ``ACK'' when he broadcast new overwrite
        value
\end{itemize}

\subsubsection{Majority Voting Algorithm}
Main idea is based on a quorum principle,
\begin{itemize}
	\item Always write to and read from a majority of nodes
	\item At least one node knows most recent value
\end{itemize}

\paragraph{Quorum Principle}
Divide the system into quorums, any two quorums should intersect.
There are different type of quorum
\begin{itemize}
	\item Majority Quorum
		\begin{itemize}
            \item[Pro]: tolerate up to $\lceil N/2 \rceil$ -1 crashes
            \item[Con]: Have to read/write $\lfloor N/2 \rfloor$ + 1values
		\end{itemize}
	\item Maekawa Quorum
		\begin{itemize}
			\item Arrange nodes in MxM grid (M=sqrt(N))
			\item Write to rows, read to columns (always overlap)
            \item[Pro]: Only need to read/write sqrt(N) nodes
            \item[Con]: Tolerate at most sqrt(N)-1 crashes
		\end{itemize}
\end{itemize}

\paragraph{Implementation}
\begin{itemize}
    \item[Read]: \begin{enumerate}
            \item Broadcast read request
                \begin{itemize}
                    \item[$\to$] receiver response with local value and seq\#
                \end{itemize}
            \item Save value from majority of nodes
            \item Return value with highest seq\#
        \end{enumerate}

    \item[Write]:
        \begin{enumerate}
            \item Broadcast v and seq\# 
                \begin{itemize}
                    \item[$\to$] receiver update to v \textcolor{red}{if newer seq\#}
                \end{itemize}
            \item Wait for ACK from majority of node
        \end{enumerate}
\end{itemize}

The problem with the first algorithm is that old write can override
new write. (resolve by \textcolor{red}{red})

\subsection{Single Storage}

\paragraph{Safety requirements}
\begin{itemize}
	\item Sequential Consistency: Only allow executions whose results
	appear as if there is a single system image and ''local time'' is
	obeyed
	\item Linearizability/Atomicity: Only allow executions whose 
	results appear as if there is a single system image and 
	''global time'' is obeyed.
\end{itemize}

\paragraph{Liveness : progress}
Liveness requirements : three progressively weaker versions
\begin{itemize}
	\item Wait-free(strongest) : Every correct node should ''make progress''
	(no deadkocks,no livelocks,no starvation)
	\item Lock-free/non-blocking : At least one correct node should 
	''make progress'' (no deadlocks,no livelock, maybe starvation)
	\item Obstruction free/solo-termination : If a single node executes
	without interference (contention) it makes progress
	(no deadlocks, maybe livelocks,maybe starvation)
\end{itemize}

\subsection{Atomic/Linearizable Registers}
\begin{itemize}
	\item Termination(Wait-freedom): If a node is correct,each read and write op
	eventually completes
	\item Linearization Points:
	\begin{itemize}
		\item \textbf{Read ops} appears as if immediately happened at all nodes
		at some time between invocation and response.
		\item \textbf{Write ops} appears as if immediately happened at all nodes
		at some time between invocation and response
		\item \textbf{Failed ops} appears as either completed at every node
		either never occured at any node
	\end{itemize}

    \paragraph{Equivalent with}
    \begin{itemize}
        \item Validity : read 
            \begin{enumerate}
                \item[IF] read not conurrent with another write or with a failed operation
                    return last value written
                \item[ELSE] return concurrent value written
            \end{enumerate}
        \item Ordering : if read $r_1$ precede read $r_2$ then write $r1$ precedes $r2$
    \end{itemize}
\end{itemize}

\paragraph{Majority voting}
There is a a problem with the majority voting as the system could appear
as non atomic. 

\subsubsection{One writer (atomic register)}

Solution to previous issues : when reading, also make a write before responding.

$\Rightarrow$ Use \textbf{causality} to enforce \textbf{atomicity}.

\begin{figure}[h]
    \centering
    \begin{tabular}{m{7cm}m{7cm}}
		\includegraphics[width=7cm]{img/maj_prob.png}&
		\includegraphics[width=6cm]{img/maj_sol.png}
	\end{tabular}
		\caption{Majority Voting Problem and Solution}
\end{figure}


\subsubsection{Multiple writer (atomic Registers)}

With multiple writer, their \#seq might be non-synchronized
which as the effect of ignoring some write operation.

\paragraph{Solution}
\begin{enumerate}
	\item Get \#seq before writing by reading from
	the majority (to get last seq num)
	\item Send Ack if receive write with old \#seq
\end{enumerate}

\paragraph{Message passing}
Message passing can be simulated using shared memory. The idea
consist of using a register AB to simulate the channel between A and B like a pipe.

\begin{itemize}
    \item $\to$ Message passing and shared memory equivalent in functionnality
but not in always in efficiency.
\end{itemize}

\subsection{Linearizability}
\subsubsection{Formalism}
\begin{itemize}
	\item $R-inv_i(X)$ Read invocation by node i on register X
	\item $R-res_i(a)$ Response with value a to read by node i
	\item $W-inv_i(X,a)$ Write invocation by node i on register
	X with value a.
	\item $W-res_i$ Response (confirmation) to write by node i
\end{itemize}

\subsubsection{Executions}
Every execution consists of:
\begin{itemize}
	\item Read operations composed of two events : 
	$R-inv_i(X)$ and $R-res_i(a)$
	\item Write operations which consist of two events:
	$W-inv_i(X,a)$ and $W-res_i$
	\item An execution is sequential if:
	\begin{itemize}
		\item X-inv by i immediately followed by a corresponding X-res at i
		\item X-res by i immediately follows a corresponding X-inv by i
		\item no concurrency, read x by p1, write y by p5,...
	\end{itemize}
\end{itemize}

\subsubsection{Assumption}
\begin{itemize}
    \item An operation O is pending in execution E if O has no response event 
    \item An execution is complete if every operation is complete (else partial)
    \item An operation X precedes operation Y in exec E if response of X is before invocation of Y in E
    \end{itemize}

%TODO End 7-share 

\section{Group Membership}

\begin{itemize}
    \item process need to know which processes are participating in the 
        computation and which are not. 

    \item Failure detectors provide such information however it's not       
        coordinated even if failure detector is perfect.                        

        (Crash detected at different time for different processes)
\end{itemize}

\begin{figure}[!h]
    \centering
    \begin{tabular}{cc}
        \includegraphics[width=7cm]{img/gm1.png}&
        \includegraphics[width=7cm]{img/gm1.png}\\
    \end{tabular}
    \caption{Group membership}
\end{figure}

\begin{itemize}
	\item Like FD, processes are informed about failures; we say that
	the processes install \textbf{views}.
	\item Like PFD, processes have accurate knowledge about failures.
	\item Unlike a PFD, the information about failures are 
	coordinated, the processes install the same sequence of view.
\end{itemize}

\subsection{Properties}
\begin{description}
	\item[Local Monotonicity] If a process install view (j,M) after 
        installing (k,N), then j > k and M $\subset$ N
	\item[Agreement] No two processes install views (j,M) and (j,M') such that
	M $\neq$ M'
	\item[Completeness] If a process p crashes, then there is an integer
	j such that every correct process eventually installs view (j,M) such that 
	p is not in M
	\item[Accuracy] If some process install a view (i,M) and p is not in M
	,then p has crash (might not be true in general case)
\end{description}

\subsection{Algorithm}
\begin{lstlisting}[caption=Group membership, mathescape]
upon event <Init> do
    view := (id:0, memb:$\pi$)
    correct := $\pi$
    wait := false

upon event <crash $p_i$> do
    correct := correct \ {$p_i$}

upon event (correct $\subset$ view.memb) AND (wait == false) do
    wait := true
    trigger <ucPropose ( view.id + 1, correct)>

upon event <ucDecided (i, m)> do
    view := (id:i, memb:m)
    wait := false
    trigger <membView view>
\end{lstlisting}

\begin{figure}[h]
		\includegraphics[width=8cm]{img/gmp_1.png}
	\caption{Group membership}
\end{figure}

\subsection{Non-blocking atomic commit}
\begin{lstlisting}
Request: <nbacPropose | v>      // propose value for the commit (0 or 1)
Indication: <nbacDecide | v>    // indique decided value
\end{lstlisting}

\begin{itemize}
    \item[NBAC1] : \textbf{Uniform agreement} : not two processes decide different values
    \item[NBAC2] : \textbf{Integrity} : no process decide two value
    \item[NBAC3] : \textbf{abort-validity} : 0 can only be decided if some process propose 0 or crashes
    \item[NBAC4] : \textbf{commit-validity} : 1 can only be decided if no process propose 0 
    \item[NBAC5] : \textbf{Termination} : every correct process eventually decide
\end{itemize}


\section{Peer-to-Peer}

\subsection{Overview P2P systems}
P2P computing is distributed computing  with the following desirable properties:
\begin{itemize}
	\item Resource sharing
	\item Dual client/server role
	\item Decentralization/autonomy
	\item Scalability
	\item Robustness/self-organization
\end{itemize}

\subsection{Distributed Hash Table}

DHT are third generation of P2P :

\begin{tabular}{m{10cm}m{6cm}}
\begin{itemize}
    \item a dynamic distribution of a hash table onto a set of cooperating nodes.
        (different node contains different parts of the table).
    \item Each node has a routing table that points to some other nodes.
    \item Lookup operation is done by node to make a key resolution
\end{itemize}
&
    \includegraphics[width=6cm]{img/DHT.png}

     \textcolor{red}{Node D: lookup(9)}\\
    \\
\end{tabular}

\begin{lstlisting}[caption=Interface DHT]
put(key,value).
get(key)
\end{lstlisting}

\subsubsection{Chord}

\begin{tabular}{m{10cm}m{3cm}m{3cm}}
\begin{itemize}
    \item \textbf{Routing table size} $= M$ s.t $N = 2^M$
    \item Every node $n$ know successor $(n + 2^{i-1})$ for $i = 1..M$
    \item $log_2(N)$ maximum hops between two node
\end{itemize}
& \includegraphics[width=3cm]{img/chord.png}
& \includegraphics[width=3cm]{img/chord1.png}\\
& \multicolumn{2}{c}{$\Rightarrow$ 1 get(15)}
\end{tabular}

\subsubsection{Generally}

\begin{itemize}
    \item We have $N$ node
    \item[$\to$] Trade-off between lookup length and routing table size

   \begin{eqnarray*}
            H &= log_k(N)\\
            R &=(k-1) H \\
            N &= (\frac{R}{H} + 1)^H\\
        \end{eqnarray*}
\end{itemize}

Chord is a special case where $k=2$.

\subsubsection{DKS}

\begin{itemize}
    \item Tunability : \begin{tabular}{l}
            Routing table size vs lookup length\\
            fault-tolerance degree
        \end{tabular}
    \item Local atomic join and leave (strong guarantees)
    \item Correction-on-use : no unnecessary bandwith consumption
\end{itemize}

\subsubsection{Design DKS(n, k, f)}
\begin{tabular}{m{10cm}m{3cm}m{3cm}}
\begin{itemize}
    \item An \textbf{identifier space} of size $N=k^L$
    \item A hash function
    \item An item ($key, value$) is stored at \textbf{successor} of
        H(key)
    \item \textbf{Bidirectional} linked list of nodes
    \item Resolving key : $\bigoh(N) hops$
\end{itemize}
& \includegraphics[width=3cm]{img/dks1.png}
& \includegraphics[width=3cm]{img/dks2.png} \\
\end{tabular}


\paragraph{Principle}

\begin{enumerate}
    \item \textbf{Distributed K-ary search}:
        \begin{itemize}
            \item resolution key : $log_k(N)$ hops
            \item At each node, a $RT$ of $log_k(N)$ level
            \item Each level of $RT$ has $k$ intervals
            \item For level $l$ ant interval $i$ : $RT(l)(i) = $ adress of the fist node
                that follow the start of the interval $i$
        \end{itemize}

        \paragraph{Interval routing}
        \begin{enumerate}
            \item If key between my predecessor and me : done
            \item Otherwise systematic forwarding level by level
        \end{enumerate}

        \paragraph{Example with $k=4$, $N=16$}: 

        \begin{tabular}{m{6cm}m{6cm}m{4cm}}

            \begin{center}\textcolor{purple}{Level 1} \end{center}
            \begin{eqnarray*}
                I_0 &= [1, 5[ &: RT(1)(0) = 1\\
                I_1 &= [5, 9[ &: RT(1)(1) = 6\\
                I_2 &= [9, 13[ &: RT(1)(2) = 10\\
                I_3 &= [13, 1[ &: RT(1)(3) = 15\\
            \end{eqnarray*}
            &
            \begin{center}\textcolor{green}{Level 2} \end{center}
            \begin{eqnarray*}
                I_0 &= [1, 2[ &: RT(2)(0) = 1\\
                I_1 &= [2, 3[ &: RT(2)(1) = 2\\
                I_2 &= [3, 4[ &: RT(2)(2) = 3\\
                I_3 &= [4, 5[ &: RT(2)(3) = 6\\
            \end{eqnarray*}

            & \includegraphics[width=4cm]{img/design1.png}
        \end{tabular}


    \item \textbf{Local atomic action for guarantees}:

        Use local atomic operation for \textbf{join, leave} to ensure
        that any key-value pair previously inserted is found despite concurrent
        joins/leaves.

        \begin{enumerate}
            \item Atomically insterted by it's current successor on the virtual space
            \item New node receive approximate routing information from it's current successor
            \item[$\to$] Concurrent join on the same segment are serialized (because of local
                atomic action)
        \end{enumerate}

        \begin{tabular}{m{10cm}m{4cm}}

            \begin{itemize}
                \item Node 14 join
                \item Pointer for node 1 on level=1 interval=3 become invalid
                \item[$\to$] corrected by correction-on-use
            \end{itemize}

            & \includegraphics[width=4cm]{img/design2.png}
        \end{tabular}

    \item \textbf{Correction-on-use}:

        \begin{tabular}{m{10cm}m{4cm}}
            Add i (interval) and l (level) with the message 
            $\Rightarrow$ $n'$ can comppute $x_i^l(n)$

            \begin{itemize}
                \item[a)] lookup(Key:13, level:1, Interval:3)
                \item[b)] badPointer(Key:13, Candidate:14)
                \item[c)] lookup(Key:13, level:1, Interval:3)
            \end{itemize}

            & \includegraphics[width=4cm]{img/design3.png}
        \end{tabular}
\end{enumerate}


\subsection{Broadcast in DHTs}

\begin{itemize}
    \item \textbf{DHT as distributed k-ary search} : construct a spanning tree
        derived from the decision tree of the distributed k-ary search after removal 
        of the virtual hops

        \paragraph{Invariant}:
        \begin{itemize}
            \item Any node sends to distinct routing entries
            \item Any sender informs receiver about a \textbf{forwarding limit}
            \item[$\to$] Construct disjoint interval and node receives a message once
        \end{itemize}

        \begin{figure}[!h]
            \centering
            \begin{tabular}{m{4cm}m{4cm}m{4cm}}
                \includegraphics[width=4cm]{img/DHT1.png}
                & \includegraphics[width=4cm]{img/DHT2.png}
                & \includegraphics[width=4cm]{img/DHT3.png}\\
            \end{tabular}
            \caption{DHD spanning tree}
        \end{figure}


    \item[Alternative]

    \item \textbf{Gnutella-like flooding in DHT} : 
        \begin{itemize}
            \item[Pro] know diameter $\to$ correct TTL $\to$ High guarantees
            \item[Con] High traffic with redundant messages
        \end{itemize}

    \item \textbf{Traversing the ring in Chord or Pastry} : 
        \begin{itemize}
            \item[Pro] No redundant message
            \item[Con] Sequential execution time
            \item[Con] Highly sensitive to failure
        \end{itemize}
\end{itemize}


\subsection{Application infrastructure in DHT}
%TODO

\section{Gossip}

Gossip is the process by which an information will spread among 
entities. Important technique to solve proble in dynamic large scale system
(scalable, simple, robust)

\begin{lstlisting}
push (Telling to)
pull (asking to)
\end{lstlisting}

\subsection{Protocol Characterisitc}
\begin{itemize}
	\item Cyclic/periodic, pair-wise interaction between peers
	\item The amount of information exchanged is of (small)
	bounded size per cycle
	\item The state each peer is bounded(small)
	\item Reflection of the state of one of both peers by the 
	change of the state of the other during interaction. 
	\item Selection of peer is random (full peer set or small set of neighbors)
	\item Reliable communication is not assumed
	\item Protocol cost is negligible
\end{itemize}

\subsection{Protocol Usage}
\begin{itemize}
	\item \textbf{Dissemenitation} : Spread information in a manner that produces
	bounded worst-case loads.
	\item \textbf{Repairing} : Anti-entropy  protocols for reparing replicated data
	, which operate by comparing replicas and reconciling differences.
	\item \textbf{Membership} : Track processes 
	\item \textbf{Aggregates} : Compute a network-wide aggregate. (Ex: number of node or 
	computing average/max/min )
	\item ...
\end{itemize}

\subsection{Information dissemination}

\begin{itemize}
\item Start with one peer that wants to disseminate some message. 
\item Then every peer does the following:

\begin{enumerate}
	\item Buffers every message (information unit) it receives up to a 
	certain buffer capacity $b$
	\item Forwards that message a limited number of hops or  time steps $t$
	\item  Forwards the message each time to frandomly selected set of 
	processes 
\end{enumerate}
\end{itemize}

\subsubsection{Infect-forever Model}
\begin{itemize}
	\item Fixed population of size n (at round 1, one is infected)
	\item If infected, stays infected forever
	\item $Y_r$ is the number of individuals infected at round r
	\item $f$ is the number of individuals that infected ones try to infect
\item $R$ is the number of round to infect all population
\end{itemize}

\begin{eqnarray*}
Y_r &\approx& \frac{1}{1+n.e^{-f.r}}\\
R &=& log_{f+1}(n)+\frac{1}{f}log(n)+O(1)\\
\end{eqnarray*}

\subsubsection{Infect-die Model}

\begin{itemize}
\item Infectious process ''remains infectious'' for just one round
\item $p_i$ is the proportion of processes eventually contaminated
\item $R$ the number of round to infect all population:
\end{itemize}

\begin{eqnarray*}
\pi &=& 1-e^{-\pi.f}\\
R &=& \frac{log(n)}{log(log(n))}+O(1)\\
\end{eqnarray*}

\subsubsection{Membership}

To ensure scalability, each process has a partial view (random sample of
node).

\begin{itemize}
	\item When a process forwards a message, it 
	includes in this message a set of processess it knows.
	\item Hence, the process that receives the message can enhance the 
	list of processes it knows by adding new processes.
\end{itemize}

\paragraph{Protocols requirement}
\begin{itemize}
\item Uniformity : all nodes play the same role
\item Adaptivity under churn : the parameter have to be tuned (t/f)
\item Bootstrapping : how do nodes enter and leave, how to start
\end{itemize}


\subsubsection{Buffer Management}

Depending on broadcast rate , buffer capacity if processes may be
insufficient to ensure that every message is buffered long enough.

\paragraph{To conter that}
message are classiied according to their
age (the number of processes the message went through) $\to$
Old message are replaced.

\subsection{Small-world network}

There are different way at which a process choose its infection target

\begin{description}
	\item[Nearest-neighbor network] Target is a neighbors (number of
	round is $\bigoh(n^{1/D})$ for a D-dimensional grid)
	\item[Random network] Target is random (number of rounds is
	$\bigoh(log(n))$)
	\item[Small world network] In between case (number of rounds
	is $\bigoh(log(n))$)
\end{description}

\paragraph{Note:} Real world social network tends to be small-world networks.


\subsubsection{Properties}

Small-world network has both nearest neighbor connection as well as 
long range connections.(Node reachable with a small number of hops).

\paragraph{Properties}
\begin{itemize}
	\item Small average shortest path length (opposed to large path length
	in Neighbor graphs)
	\item A high clustering coefficient (opposed to low CC in Random graphs)
	\begin{description}
	\item[Clustering coefficient] = Average of 

        $$\forall_{p_i \in nodes} \frac{\sum { edge | \quad edge.start \in
            p_i.neighbors \quad AND \quad edge.end \in p_i.neighbors }}
            {p_i.nbrNeighbors (p_i.nbrNeighbors-1) /2)}$$

\end{description}
\end{itemize}

The clustering coefficient CC measures degree of clustering. Count the 
number of edges between neighboring nodes and divide by the maximum
possible for every node. The average of this number is the CC.

\subsection{Gossip Framework}
\begin{tabular}{m{0.5\linewidth}m{0.5\linewidth}}
\begin{lstlisting}
// active thread 
do forever 
    wait(T time units) 
    q = SelectPeer() 
    push S to q 
    pull Sq from q 
    S = Update(S,Sq) 
\end{lstlisting}
&
\begin{lstlisting}
// passive thread 
do forever 
    (p,Sp) = pull * from * 
    push S to p 
    S = Update(S,Sp) 
\end{lstlisting}
\end{tabular}

To instantiate the framework, define 
\begin{itemize}
	\item Local state \texttt{S},
	\item Method \texttt{SelectPeer()} and \texttt{Update()}
	\item Style of interaction : \texttt{push, pull, push-pull}
\end{itemize}

\subsubsection{Aggregation}
The style of interaction is push-pull
\begin{itemize}
	\item S is the current estimate of global aggregate
	\item \texttt{SelectPeer()}: Single random neighbor
	\item \texttt{Update()}: Numerical function defined according to desired global
	aggregate(arihtmetic/geometric mean,max,\ldots)
\end{itemize}
$$ p = \frac{E(\phi^2_{i+1})}{E(\phi^2_i)} \approx \frac{1}{2 \sqrt{e}} \approx 0.303 $$

%TODO Network-size

\subsubsection{Topology Management}
Topology gradually appears as a result of a ranking function


\begin{itemize}
\item Given a set of N node (each node has a \texttt{view} of size c, \texttt{profile} is used
to calculate ranking)
\item Given a ranking function R : $R(x, {y_1,...,y_m}) = {all ordering of {y_1,...,y_m}}$
\begin{itemize}
\item[$\to$] R can be defined by a distance function
\end{itemize}
\end{itemize}

\paragraph{T-Man algorithm}
\begin{itemize}
    \item \texttt{InitialView()} : random sample of nodes
    \item \texttt{Merge($v_1, v_2$)} : $v_1 \cup v_2$
    \item \texttt{SelectPeer(v)} : rank current view v with R and return
    random sample from first half
    \item \texttt{SelectView(b)} : rand b with R and return first c element
\end{itemize}

\paragraph{Convergenge}
Rapide convergence phase followed by endgame phase.
\begin{itemize}
\item After 1 cycle, view is closest c out of $2c$
\item After i cycle, view is closest c out of $2^i c$
\end{itemize}

\subparagraph{Handling the endgame}
\begin{itemize}
\item Balancing : during rapid convergence, receiving node refuses
contact if there are already too many
\item Routing : ing endgame, instead of random selection of a
peer node, select closest one on the topology (with
exponentially decreasing probabilities)
\end{itemize}

\subsubsection{Heartbeat Synchronization}


\section{Bit torrent}

\paragraph{Problem}
\begin{itemize}
\item Large number of user $\bigoh(10^3)$
\item Fast as possible
\end{itemize}

\subsection{Naïve solution}
All client download from the server with the maximum
number of parallel client (depend of the available bandwith).

\paragraph{Problem}
It's really slow and to sequential ordering.
In addition, we don't use upload resources of clients.

\subsection{Pieces}

A file is sliced into a number of smaller transfer units called pieces of a
small predefined size.

\begin{itemize}
\item \textbf{Selecting the right peers} : From the point of view of one peer, selecting the
right peer to connect to can make a difference in the speed of downloading a file
\item \textbf{Selecting the right pieces} : Not only peer selection is important, piece
selection can affect the download time
\item[$\Rightarrow$] maximize the utilization of network resource of everybody
\end{itemize}

\subsubsection{Select peers and pieces}

\begin{itemize}
\item Global knowledge OR no global knowledge
\item[$\to$] In real P2P, the bandwith changes over time, peers come and go,
peers are selfish (Download file and run away)

\item[Solution] : random decision s.t 
\begin{itemize}
\item Make sur that all peers have roughly the same number of links
\item A peer choose random piece from the peers he is connected
\end{itemize}
\end{itemize}

\subsection{Bittorent}

\subsubsection{Terminology}
\begin{description}
\item[Seeds] : machines that have a complete copy of the file. They are usually selfish and do not want to
wait after they get the file.

At least the first seed has to stay to serve
one complete copy of the file.

\item[Leechers] : Any peer who does not have a complete file
is called a leecher
\item[Tracker] : A peer that keeps track of who are the seeds and the leechers.
\item[Tracker.getPeers] : Give a subset of peers
\item[.torrent file/Meta-info File] : \includegraphics[width=7cm]{img/torrent.png}
\item[Handshaking] : Once a peer list is received, a handshake message
is sent to the all peers
\item[Exchanging bitfields] after handshaking with peers

\end{description}

\begin{figure}[!h]
\centering
\includegraphics[width=9cm]{img/after.png}
\caption{After exchange}
\end{figure}

%TODO Choking

\subsubsection{Piece selection}
\begin{itemize}
\item Rarest-first amoung your peers
\item Random-first piece to get first piece as fast as possible (only first few piece)
\item EndGame mode 
\end{itemize}

\subsubsection{Peer selection}
\begin{enumerate}
\item Construction of the peer list by using the tracker
\item Unchoking (=Choice based on which peer can give me pieces faster)

\begin{itemize}
    \item A peer has a total of 80 connections, 40 initiated by
    him and 40 by others
    \item A peer unchokes only 4 peers simultaneously
    \item Decision about unchoking is reconsidered every ten
    seconds
    \begin{itemize}
        \item 3 peers are unchoked (given that they are interested)
        based on their upload rate to me
        \item Optimistically look for a fast guy randomly in the 40
        remaining connections
    \end{itemize}
\end{itemize}
\end{enumerate}


\section{Raft}

\begin{itemize}
    \item At any given time, each servier is EITHER :
        \begin{enumerate}
            \item Follower : completely passive 
            \item Candidate : used to elect a new leader
            \item Leader
        \end{enumerate}
    \item Time divided into term (maintains current term value)
\end{itemize}

\subsection{Leader election}
\begin{enumerate}
    \item Increment current term
    \item Change to candidate state
    \item Vote for self
    \item Send request vote RPC to all other server
        \begin{itemize}
            \item Receive vote from majority $\Rightarrow$ leader
            \item Receive RPC from valid leader $\Rightarrow$ follower state
            \item Ellection timeout elapse $\Rightarrow$ new election
        \end{itemize}
\end{enumerate}

\subsection{Log replication}
Leader take commands from clients and replicates its log to other server.

\begin{thebibliography}{1} 
\bibitem{icampus} http://www.icampus.uclouvain.be, {\em Icampus}
\end{thebibliography}

\end{document}
